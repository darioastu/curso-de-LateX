 \documentclass[a4paper,onecolumn,12pt]{article}
 %paquetes
  \usepackage[spanish]{babel}
  \usepackage[total={18cm, 21cm}, top=2cm, left=2cm]{geometry}
  \usepackage[utf8]{inputenc}
  \usepackage{graphicx}
  \usepackage{color}
  %comandos
  \parindent = 25mm
  \author{Dario Astudillo }
  \title{Mi Primer Latex}
  \date{Today}
  %contenido
  
\begin{document}
	\maketitle

	Mi primer  LaTeX
	compilar y mirar el documento
	guardar
	\section[epn]{Escuela Politécnica Nacional}
	\subsection{calculo en una variable}
	\subsubsection{derivadas}
	\textbf{como calcular }
	
	\texttt{programación}
	
	{\normalsize cogito,ergo sum}
	\emph{Algo de texto negro, \color{red}
seguido por un fragmento rojo}, {\color
{blue} finalmente algo de texto azul.}
 
   usar colores  oscuros para poder ver 
   usar colores  oscuros para poder ver usar colores 
    oscuros      para poder   colores  oscuros para poder ver
     
   \qquad  LATEX\qquad inicia inmediatamente un nuevo renglón (sin sangría), insertando un espacio
vertical de longitud dada, antes del nuevo renglón. El texto que precede a esta instrucción
no es justificado na la derecha. El argumento [Longitud] es opcional; es decir, con \\ simplemente
se inicia un nuevo renglón (sin sangría).
\\[10mm]
   LATEX inicia inmediatamente un nuevo renglón (sin sangría), insertando un espacio
vertical de longitud dada, antes del nuevo renglón. El texto que precede a esta instrucción
no es justificado na la derecha. El argumento [Longitud] es opcional; es decir, con \\ simplemente
se inicia un nuevo renglón (sin sangría).
\\[10mm]
 


\end{document}