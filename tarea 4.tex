%%%%%%%%%%%%%%%%%%%%%%%%%%%%%%%%%%%%%%%%%

%----------------------------------------------------------------------------------------
%       Clase, paquetes y configuraciones
%----------------------------------------------------------------------------------------

\documentclass[11pt, a4paper]{article} % Font size

\usepackage[utf8]{inputenc}
\usepackage[T1]{fontenc}
\usepackage[spanish,es-nolayout,es-nodecimaldot,es-tabla]{babel}
\usepackage{amsmath}
\usepackage{amsfonts}
\usepackage{amssymb,amsthm}
\usepackage{enumerate}
\usepackage{enumitem}
\usepackage{parskip}
\usepackage{nicefrac}
\usepackage[left=2cm,right=2cm,top=2.5cm,bottom=2cm]{geometry}
\usepackage[colorlinks = true]{hyperref}
\usepackage{listings}

%
\newtheorem{teo}{Teorema}

% Comandos
\newcommand{\R}{\mathbb{R}}
\newcommand{\yds}{\qquad\text{y}\qquad}
\DeclareMathOperator{\proy}{proy}
\DeclareMathOperator{\dd}{d}

\linespread{1.25}

%----------------------------------------------------------------------------------------
%       Datos informativos
%----------------------------------------------------------------------------------------
\title{\begin{large}CURSO DE LATEX\end{large}\\ Tarea 4}
\author{Dario Astudillo}
\date{\today}   

% Contenido
\begin{document}
	\maketitle
\begin{itemize}
\item\bf{Programa dia de la semana:}
\\
Generar un programa que al ingresas un numero de 1 al 7 indique el dia de la semana correspondiente.
\end{itemize}

 
\begin{lstlisting}[frame=single]

//PROGRAMA//ELECTRONICA  Y TELECOMUNICACIONES//DARIO ASTUDILLO

#include<stdio.h>
#include<conio.h>
#include<iostream.h>

int main()

{

    int dia ;


    printf(" \n\            PROGRAMA DIA DE LA SEMANA\n\n ");

    printf(" \n\ingrese un numero de 1 al 7 \n\n ");
    scanf("%d",&dia);
    switch(dia)
    {
		case 1:
	 printf(" \n\el dia es lunes\n\n ");
		break;
      case 2:
	 printf(" \n\el dia es martes \n\n ");
		break;
      case 3:
	 printf(" \n\el dia es miercoles \n\n ");
		break;
      case 4:
	 printf(" \n\el dia es jueves \n\n ");
		break;
      case 5:
	 printf(" \n\el dia es viernes\n\n ");
		break;
      case 6:
	 printf(" \n\el dia es sabado \n\n ");
		break;
      case 7:
     printf(" \n\el dia es domingo \n\n ");
		break;
       default:
             printf("no existe ese dia \n");
             break;

          }

    getch();

}
\end{lstlisting}

\end{document}